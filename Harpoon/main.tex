\documentclass[letterpaper,twocolumn,openany,nodeprecatedcode]{dndbook}

% Use babel or polyglossia to automatically redefine macros for terms
% Armor Class, Level, etc...
% Default output is in English; captions are located in lib/dndstring-captions.sty.
% If no captions exist for a language, English will be used.
%1. To load a language with babel:
%	\usepackage[<lang>]{babel}
%2. To load a language with polyglossia:
%	\usepackage{polyglossia}
%	\setdefaultlanguage{<lang>}
\usepackage[english]{babel}
%\usepackage[italian]{babel}
% For further options (multilanguage documents, hypenations, language environments...)
% please refer to babel/polyglossia's documentation.

\usepackage[utf8]{inputenc}
\usepackage[singlelinecheck=false]{caption}
\usepackage{lipsum}
\usepackage{listings}
\usepackage{shortvrb}
\usepackage{stfloats}

\captionsetup[table]{labelformat=empty,font={sf,sc,bf,},skip=0pt}

\MakeShortVerb{|}

\lstset{%
  basicstyle=\ttfamily,
  language=[LaTeX]{TeX},
  breaklines=true,
}

\title{DnD Homebrew}
\author{Felix Springer}
\date{2022/09/05}

\begin{document}

\frontmatter

\maketitle

\tableofcontents

\mainmatter%

\chapter{Piratenabenteuer}

\section{Weapons}

\subsection{Harpoon}

The harpoon is a weapon used for hunting whales.
Watch out for the shot if you can hear someone shout "There she blows!".
\begin{DndTable}{XX}
    \textbf{Property}  & \textbf{Value} \\
    Cost & 75 gp \\
    Damage & 1d8 piercing \\
    Weight & 12 lbs \\
    Properties & Two-Handed, Range (40/80), Ammunition (1), Loading, Special
\end{DndTable}

When you attack with the harpoon, a spear-like arrow is shot from the harpoon.

The arrow is attached to the harpoon by hempen rope reinforced with metal wires.
The rope uses a spring mechanism, that automatically pulls the spear back, when there is no resistance to it.
This force is not able to move any targets, but it reloads the harpoon when a shot misses or the spear is removed from a target.
A hit target can move freely in the circle around the shooter with the shooting distance as a radius.
When the target is closing in on the shooter, the circle becomes smaller.

If a creature tries to run out of this circle, while attached with the rope, that creature must make a DC 14 saving throw.
On a successful saving throw the creature drags the shooter around.
On failure the creature takes 1d4 piercing damage and is still attached to the harpoon.

A hit target can use its action to pull out the spear by making a DC 12 Strength check.
On success the creature takes 2d6 slashing damage and the spear is dropped to the ground.
On failure the creature takes 1d4 piercing damage and is still attached to the harpoon.

% The damages and DCs are just what came to my mind.
%
% Maybe additionally:
% The connection can be cut by the target or any other creature next to the rope.
% To do this, the creature must deal 20 slashing damage to the rope (AC 10).

\end{document}
